\label{features}
\label{features}
For each code sample, we calculate the persistent homology in dimensions $0$ and $1$ of 
the symmetrized attention matrices, obtaining the persistence diagram for each 
attention head of the BERT model. We compute the following features in each dimension 
from the diagrams:
\begin{itemize}
    \item Mean lifespan of points on the diagram
    \item Variance of the lifespan of points on the diagram
    \item Maximum lifespan of points on the diagram
    \item Total number of points on the diagram
    \item Persistence entropy
\end{itemize}

We symmetrize attention matrices to enable the application of persistent homology techniques. 
Symmetrizing attention matrices allows us to interpret them as distance matrices of a point cloud embedded in Euclidean space. 
We symmetrize attention matrices as follows: 
\begin{equation}
    \forall i,j: W^\mathrm{sym}_{ij} = \max{(W_{ij}^\mathrm{attn}, W_{ji}^\mathrm{attn})}.
\end{equation}

Alternatively, one can think of attention graphs, in which an edge between the vertices $i$ and $j$
appears if the threshold is greater than both $W_{ij}^\mathrm{attn}$ and $W_{ji}^\mathrm{attn}$.

We consider these features asthe numerical characteristics of the semantic evolution processes 
in the attention heads. These features encode the information about the clusters of mutual influence
of the tokens in the sentence and the local structures like cycles. The features with <<significant>> 
persistence (i.e. those with large lifespan) correspond to the stable processes, whileas the features
with short lifespans are highly susceptible to noise and do not reflect the stable topological attributes.
