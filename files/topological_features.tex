\label{features}
For each code sample, we calculate the persistent homology in dimension $0$ and $1$ 
of the symmetrized attention matrices, obtaining the persistence diagram on each 
attention head of the BERT model. We compute the following features in each dimension 
from the diagrams:
\begin{itemize}
    \item The mean lifespan of points on the diagram
    \item The variance lifespan of points on the diagram
    \item The max lifespan of points on the diagram
    \item The overall number of points on the diagram
    \item The persistence entropy
\end{itemize}

We symmetrize attention matrices in order to be able to use the persistent homology
machinery. When attention matrices are symmetrized, one can think of attention matrices as
the distance matrices of some point cloud embedded in Euclidian space. We symmetrize attention
matrices in the following manner: 
\begin{equation}
    \forall i,j: W^\mathrm{sym}_{ij} = \max{(W_{ij}^\mathrm{attn}, W_{ji}^\mathrm{attn})}.
\end{equation}

Alternatively, one can think of attention graphs, in which an edge between the vertices $i$ and $j$
appears when the threshold is greater than both $W_{ij}^\mathrm{attn}$ and $W_{ji}^\mathrm{attn}$.

We consider these features as the numerical characteristic of the semantic evolution processes 
in the attention heads. These features encode the information about the clusters of mutual influence
of the tokens in the sentence and the local structures like cycles. The features with <<significant>> 
persistence (i.e. those with large lifespan) correspond to the stable processes, whileas the features
with short lifespan are highly influenced to noise and doesn't reflect the stable topological attributes.
