\documentclass[runningheads]{llncs}
\usepackage{graphicx} % Required for inserting images
\usepackage[T1]{fontenc}
\usepackage[utf8x]{inputenc}
\usepackage[english]{babel}
\usepackage[title]{appendix}


\usepackage{amsmath}
\usepackage{amssymb}
\usepackage{mathtools}


\makeatletter
\newcommand{\@chapapp}{\relax}%
\makeatother

\def\keywordname{{\bf Keywords:}}%

\title{Vulnerability Detection via Topological Analysis of Attention Maps}
\author{Snopov P. \inst{1},
Golubinsky A.N.\inst{1}}

\institute{Institute for Information Transmission Problems of Russian Academy of Sciences
%Институт проблем передачи информации им. А.А. Харкевича РАН, 127051, г.Москва, Б. Каретный пер., д.19, стр.1 \and
\email{snopov@iitp.ru}\\
}

\begin{document}

\maketitle
\begin{abstract}
    Recently, deep learning (DL) approaches to vulnerability detection have gained significant traction. These methods demonstrate promising results, often surpassing traditional static code analysis tools in effectiveness.

    In this study, we explore a novel approach to vulnerability detection utilizing the tools from topological data analysis (TDA) on the attention matrices of the BERT model. Our findings reveal that traditional machine learning (ML) techniques, when trained on the topological features extracted from these attention matrices, can perform competitively with pre-trained language models (LLMs) such as CodeBERTa. This suggests that TDA tools, including persistent homology, are capable of effectively capturing semantic information critical for identifying vulnerabilities.

    
    \keywords{Vulnerability detection, Persistent homology, Large language models}
    \end{abstract}

\section{Introduction}
The problem of the source code vulnerability detection becomes an important problem in the field
of software engineering nowadays. This is a complex task involving the analysis of both syntax
and semantics of the source code. The static analysis methods were dominating during most of the time
in the vulnerability detection. The static tools tend to rely heavily only on the syntax of the program,
thus loosing the potential information that is hidden in the semantics. Therefore such approaches
suffer from high rates of false positives.

But recently the different ML-based and DL-based methods started to
appear \cite{sharma_survey_2024, steenhoek_empirical_2023}. 
Most DL-based solutions provide the methods based on graph neural networks (GNNs) \cite{wen_vulnerability_2023, nguyen_regvd_2022, zhou_devign_2019}.
Besides GNN-based models, there exists LLM-based approaches \cite{shestov_finetuning_2024, steenhoek_comprehensive_2024, guo_graphcodebert_2021}. 
Both GNN-based and LLM-based methods are targeted to learn the semantics together with the syntax 
of the program. This approaches show the promising results, in particular such methods demonstrate 
low rates of false positives as well as false negatives \cite{katsadouros_survey_2022}.

However some studies show that the generalization of DL-based methods is very poor, and that 
there are no existing models that would perform well in real-world settings \cite{saikat_chakraborty_deep_2022}. 
This situation shows that there is still no silver bullet for the vulnerability detection task. 

Our work demonstates a novel approach to the vulnerability detection problem. As mentioned above,
the superiority of DL-based models over static ones are potentially due to the use of the semantic
information of the program. This semantic information is captured somehow with neural networks 
during training or fine-tuning. Moreover, LLM-based models should capture this information even
wihout any fine-tune.

On the hand, as well as with the syntax of the source code, the semantics can also be represented
as a tree or a graph. Such representation opens the doors for the use of topological methods.

In this work, we leverage the BERT model, pretrained on the source code, for capturing 
the semantic information of the programs, and we also use the tools from topological data 
analysis \cite{something-1} to retrieve the relations in this semantics. 
This is an attempt to analyse the interpret the attention matrices via topological instruments.

Our work is 
inspired by the works of Laida Kushareva et. al. \cite{kushnareva_artificial_2021}, in which the 
authors apply the topological methods for the artificial text detection, and show that the topology
of the semantics captured by a LLM model (such as BERT) provides enough information for the 
successful classification, beating neural baselines and performing on par with a fully fine-tuned
BERT model, but being more robust towards the unseen data.

%\section{Related Work}
%This work resembles \cite{kushnareva_artificial_2021} and uses its techniques for the vulnerability detection task. This work is an attempt to apply TDA methods over the transformer model’s attention maps and interpret topological features for the NLP field. This is the first work applying TDA methods for the vulnerability detection task.

Another works on vulnerability detection via LLMs and GNNs.

By our knowledge, this is the first work that leverages the methods from topological data analysis
in the vulnerability detection task.
%This work resembles \cite{kushnareva_artificial_2021} and uses its techniques for the vulnerability detection task. This work is an attempt to apply TDA methods over the transformer model’s attention maps and interpret topological features for the NLP field. This is the first work applying TDA methods for the vulnerability detection task.
\section{Background}
\subsection{BERT Model}
BERT is a transformer-based language model that has pushed 
state-of-the-art results in many NLP tasks. The BERT architecture
comprises $L$ encoder layers with $H$ attention heads in each layer.
The input of each attention head is a matrix $X$ consisting of the
$d$-dimensional representations (row-wise) of $m$ tokens, so that 
$X$ is of shape $m\times d$. The head outputs an updated 
representation matrix $X^{\textrm{out}}$: 
\begin{equation}
    X^\mathrm{out} = W^\mathrm{attn}(XW^V), \textrm{ with } 
    W^\mathrm{attn} = \mathrm{softmax}\bigl( \frac{(XW^Q)(XW^K)^T}{\sqrt{d}} \bigr)
\end{equation}
where $W^Q, W^K, W^V$ are trained projection matrices of shape $d\times d$
and $W^\mathrm{attn}$ is of shape $m\times m$ matrix of attention weights.
Each element $w_{ij}^\mathrm{attn}$ can be interpreted as a weight of the
$j$-th input's {\it relation} to the $i$-th output: larger weights mean 
stronger connection between the two tokens.

\subsection{Attention Graph}
An {\it attention graph} is a weighted graph representation of an attention matrix $W^{attn}$, 
in which the vertices represent the tokens and the edges connect a pair of tokens if the 
corresponding weight is higher than the predefined threshold value. 

This threshold value is set to distinguish weak and strong relations between tokens. 
But the choice of the threshold value seems to be a very hard problem. 
Moreover, varying the threshold, the graph structure may change dramatically. 

Hopefully, TDA methods can extract the properties of graph structure, without specifying the
concrete value of the threshold.

\subsection{Topological Data Analysis}
Topological data analysis is a young and rapidly evolving field that applies some of the 
powerful methods from algebraic topology to data science. There are already exist a plethora of
good tutorials and surveys for non-mathematicians \cite{murugan2019introductiontopologicaldataanalysis,tda4ds}
as well as for those who has a mathematical background \cite{Edelsbrunner2008,Oudot2015,Schenck2022}.

The main instrument in topological data analysis, {\it persistent homology}, 
allows one to track the changes in the topological structure for different objects, such as
point clouds, scalar functions, images, weighted graphs \cite{adams2016persistenceimagesstablevector,Aktas2019}. 

Specifically, in our work, we are given with a set of tokens $V$ and an attention matrix $W$.
Next, we build a family of attention graphs, which is being indexed by increasing threshold values
This family is called a {\it filtration}, and it is a crucial object in TDA. 

When such seqeunce of graphs is given, the persistent homology in dimension $0$ and $1$ are computed.
Dimension $0$ illustrates if there are connected components or clusters in our data, and 
dimension $1$ shows if our data possesses cycles or <<loops>>. 
These calculations provide us with the {\it persistence diagram}, which can be further used to
derive specific topological features, such as the amount of connected components or the amount of
cycles (such features are also called the {\it Betti numbers}).

\section{Topological Features of the Attention Graphs}
\label{features}
For each code sample, we calculate the persistent homology in dimension $0$ and $1$ 
of the symmetrized attention matrices, obtaining the persistence diagram on each 
attention head of the BERT model. We compute the following features in each dimension 
from the diagrams:
\begin{itemize}
    \item The mean lifespan of points on the diagram
    \item The variance lifespan of points on the diagram
    \item The max lifespan of points on the diagram
    \item The overall number of points on the diagram
    \item The persistence entropy
\end{itemize}

We symmetrize attention matrices in order to be able to use the persistent homology
machinery. When attention matrices are symmetrized, one can think of attention matrices as
the distance matrices of some point cloud embedded in Euclidian space. We symmetrize attention
matrices in the following manner: 
\begin{equation}
    \forall i,j: W^\mathrm{sym}_{ij} = \max{(W_{ij}^\mathrm{attn}, W_{ji}^\mathrm{attn})}.
\end{equation}

Alternatively, one can think of attention graphs, in which an edge between the vertices $i$ and $j$
appears when the threshold is greater than both $W_{ij}^\mathrm{attn}$ and $W_{ji}^\mathrm{attn}$.

We consider these features as the numerical characteristic of the semantic evolution processes 
in the attention heads. These features encode the information about the clusters of mutual influence
of the tokens in the sentence and the local structures like cycles. The features with <<significant>> 
persistence (i.e. those with large lifespan) correspond to the stable processes, whileas the features
with short lifespan are highly influenced to noise and doesn't reflect the stable topological attributes.

\section{Experiments}
\subsubsection*{Methodology} 
To evaluate whether the encoded topological information can be used for 
vulnerability detection, we train Logistic Regression, Support Vector Machine (SVM), 
and Gradient Boosting classifiers on the topological features derived from the 
attention matrices of the BERT model, as described in Section \ref*{features}. 
We utilize the {\it scikit-learn} library \cite{scikit-learn} for Logistic Regression 
and SVM, and the {\it LightGBM} library \cite{ke2017lightgbm} for Gradient Boosting. 
Detailed training procedures are outlined in Appendix \ref*{training_details}.

\subsubsection*{Data} 
We train and evaluate our classifier on {\it Devign} dataset.
This dataset comprises samples from two large, widely-used open-source C-language projects: QEMU, and FFmpeg, 
which are popular among developers and diverse in functionality.
Due to computational constraints, we were only using those data samples, that, 
being tokenized, are of length less than $150$. This ensures that the point cloud 
constructed during attention symmetrization is also limited to a maximum length of $150$.

\subsubsection*{Baselines} 
We employ the \texttt{microsoft/codebert-base} model \cite{feng2020codebert} from the 
HuggingFace library \cite{huggingface} as our pre-trained BERT-based baseline. 
Additionally, we fully fine-tune the \texttt{microsoft/codebert-base} model for comparison.
\section{Results and Discussion}
Table \ref*{results} outlines the results of the vulnerability detection experiments 
on the {\it Devign} dataset. The results reveal that the proposed topology-based 
classifiers outperform the chosen large language model (LLM) without fine-tuning but 
perform worse than the fine-tuned version.

\begin{table}
    \label{tab:results}
    \centering
    \caption{The results of the vulnerability detection experiments.}\label{results}
    \begin{tabular}{|c|c|c|}
    \hline
    {\bf Model} & {\bf F1 score} & {\bf Accuracy} \\
    \hline
    Logistic Regression & 0.22 & 0.54 \\
    \hline
    LightGBM & {\bf 0.55} & 0.63 \\
    \hline
    SVM & 0.54 & {\bf 0.65} \\
    \hline
    CodeBERTa (pre-trained) & 0.28 & 0.45 \\
    \hline
    CodeBERTa (fine-tuned) & {\bf 0.71} & {\bf 0.72} \\
    \hline
    \end{tabular}
\end{table}


These observations indicate that the information about a code snippet's vulnerability 
is encoded in the topological attributes of the attention matrices. The semantic 
evolution in the attention heads reflects code properties that are crucial for the 
vulnerability detection task, and persistent homology proves to be an effective method 
for extracting this information.

Notably, only the semantic information from attention heads was used. 
The inclusion of additional topological features obtained from the structural 
information of the source code, such as the topology of graph representations of the 
source code, could potentially enhance the overall performance of the proposed models.
\section{Consclusion}
This paper introduces a novel approach for the vulnerability detection task based on TDA. 


\bibliographystyle{splncs04}
\bibliography{biblio}

\begin{subappendices}
    \newpage
    \renewcommand{\thesection}{\Alph{section}}%
    \section{Training Details}
    \label{training_details}
    \subsubsection*{Fine-tuning CodeBERTa} We trained with the cosine scheduler 
    with an initial learning rate $\mathrm{lr}=5e-5$ and set the number of epochs $\mathrm{e}=15$.
    \subsubsection*{Hyperparameter search} We employed {\it Optuna} \cite{optuna} to 
    find optimal hyperparameters for both SVM and LightGBM models.
    
    For SVM, the optimal hyperparameters were
    $C=9.97$, $\gamma=\mathrm{auto}$ and $\mathrm{kernel}=\mathrm{rbf}$. 
    
    For LightGBM, the optimal parameters were
    $\lambda_{\ell_1}=5.97$, $\lambda_{\ell_2}=0.05$, $\mathrm{num\_leaves}=422$, $\mathrm{feature\_fraction}=0.65$,
    $\mathrm{bagging\_fraction}=0.93$, $\mathrm{bagging\_freq}=15$, $\mathrm{min\_child\_samples}=21$.
    \section{Persistent Homology}
    \label{persistent_homology}
    Recall that a simplicial complex $X$ is a collection of $p$-dimensional simplices, 
    i.e., vertices, edges, triangles, tetrahedrons, and so on. Simplicial complexes 
    generalize graphs, which consist of vertices ($0$-simplices) and edges 
    ($1$-simplices) and can represent higher-order interactions.
A family of increasing simplicial complexes

\[
\emptyset \subseteq X_0 \subseteq X_1 \subseteq \cdots \subseteq X_{n-1} \subseteq X_n
\]

is called a \textit{filtration}.

The idea of \textit{persistence} involves tracking the evolution of simplicial complexes over the filtration. 
{\it Persistent homology} allows to trace the changes in homology vector 
spaces\footnote{Usually, homology groups are considered with integral coefficients, but in the realm of 
persistence, homology groups are taken with coefficients in some field, for example, $\mathbb{Z}_p$ for some large prime number $p$. Hence, the homology groups become homology vector spaces.} of simplicial complexes 
that are present in the filtration. Given a filtration $\{X_t\}_{t=0}^n$, the homology functor $H_p$ applied 
to the filtration generates a sequence of vector spaces $H_p(X_t)$ and maps $i_*$ between them

\[
H_p(X_*): H_p(X_0) \xrightarrow{i_0} H_p(X_1) \xrightarrow{i_1} \cdots \xrightarrow{i_{n-1}} H_p(X_n).
\]

Each vector spaces encodes information about the simplicial complex $X$ and its 
subcomplexes $X_i$. For example, $H_0$ generally encodes the connectivity of the space 
(or, in data terms, $H_0$ encodes the clusters of data), $H_1$ encodes the presence of 1-cycles, i.e., loops, 
$H_2$ represents the presence of 2-cycles, and so on. Persistence tracks the generators of each 
vector space through the induced maps. Some generators will vanish, while others will persist. 
Those that persist are likely the most important, as they represent features that truly exist in $X$. 
Therefore, persistent homology allows one to gain information about the underlying topological space via 
the sequence of its subspaces, the filtration.

In algebraic terms, the sequence of vector spaces $H_p(X_t)$ and maps $i_*$ between them can be seen as a 
representation of a quiver $I_n$. From the representation theory of quivers 
it is known, due to Gabriel, that any such representation is isomorphic to a direct sum of indecomposable 
interval representations $I[b_i, d_i]$, that is,

\[
H_p(X_*) \simeq \bigoplus_i I[b_i, d_i].
\]

The pairs $(b_i, d_i)$ represent the persistence of topological features, where $b_i$ denotes the time of 
birth and $d_i$ denotes the time of death of the feature. These pairs can be visualized via {\it barcodes} 
where each bar starts at $b_i$ and ends at $d_i$.


This information is also commonly represented using {\it persistence diagrams}. A persistence diagram is a 
(multi)set of points in the extended plane $\overline{\mathbb{R}^2}$, which reflects the structure of persistent homology.
 Given a set of pairs $(b_i, d_i)$, each pair can be considered as a point in the diagram with coordinates $(b_i, d_i)$. Thus, a 
 persistence diagram is defined as

\[
\mathrm{dgm}(H_p(X_*)) = \{ (b_i, d_i) : I[b_i, d_i] \text{ is a direct summand in } H_p(X_*) \}.
\]

Persistence diagrams provide a detailed visual representation of the topology of 
point clouds. However, integrating them into machine learning models presents 
significant challenges due to their complex structure. To effectively use the 
information from persistence diagrams in predictive models, it is crucial to 
transform the data into a suitable format, such as by applying persistence entropy.

{\it Persistence entropy} is a  specialized form of Shannon entropy specially designed for persistence
diagrams and is calculated as follows:

\[
PE_k(X) \coloneqq -\sum_{(b_i, d_i) \in D_k} p_i \log(p_i),
\]

where

\[p_i = \frac{d_i - b_i}{\sum_{(b_i, d_i) \in D_k} (d_i - b_i)} \quad \text{and} \quad D_k \coloneqq \mathrm{dgm}(H_k(X_\bullet)).\]

This numerical characteristic of a persistence diagram has several advantageous 
properties. Notably, it is stable under certain mild assumptions. This stability means 
there is a bound that <<controls>> the perturbations caused by noise in the input data.
\end{subappendices}

\end{document}
